%%%%%%%%%%%%%%%%%%%%%%%%%%%%%%%%%%%%%%%%%%%%%%%%%%%%%%%%%%%%%%%%%%%%%%%%
%%%%%%%%%%%%%%%%%%%%%% Simple LaTeX CV Template %%%%%%%%%%%%%%%%%%%%%%%%
%%%%%%%%%%%%%%%%%%%%%%%%%%%%%%%%%%%%%%%%%%%%%%%%%%%%%%%%%%%%%%%%%%%%%%%%

%%%%%%%%%%%%%%%%%%%%%%%%%%%%%%%%%%%%%%%%%%%%%%%%%%%%%%%%%%%%%%%%%%%%%%%%
%% NOTE: If you find that it says                                     %%
%%                                                                    %%
%%                           1 of ??                                  %%
%%                                                                    %%
%% at the bottom of your first page, this means that the AUX file     %%
%% was not available when you ran LaTeX on this source. Simply RERUN  %%
%% LaTeX to get the ``??'' replaced with the number of the last page  %%
%% of the document. The AUX file will be generated on the first run   %%
%% of LaTeX and used on the second run to fill in all of the          %%
%% references.                                                        %%
%%%%%%%%%%%%%%%%%%%%%%%%%%%%%%%%%%%%%%%%%%%%%%%%%%%%%%%%%%%%%%%%%%%%%%%%

%%%%%%%%%%%%%%%%%%%%%%%%%%%% Document Setup %%%%%%%%%%%%%%%%%%%%%%%%%%%%

% Don't like 10pt? Try 11pt or 12pt
\documentclass[8pt]{article}
\RequirePackage[T1]{fontenc}

% LaTeX will typeset using Computer Modern Roman, which a lot of
% non-mathematicians and non-engineers won't like. Also, a few PDF
% viewers may not render CMR very well. Instead, Times New Roman can
% be used. That's what this package does.
\usepackage{times}
\usepackage{tabu}
\usepackage{fontawesome}
% The automated optical recognition software used to digitize resume
% information works best with fonts that do not have serifs. This
% command uses a sans serif font throughout. Uncomment both lines (or at
% least the second) to restore a Roman font (i.e., a font with serifs).
% (NOTE: This requires the times package above)
%\renewcommand{\familydefault}{\sfdefault}

% This is a helpful package that puts math inside length specifications
\usepackage{calc}

% This package helps LaTeX auto-hyphenate hyphenated words if you use
% special hyphens. For example, bio\-/mimicry will properly hyphenate
% ``mimicry'' if necessary.
\usepackage[shortcuts]{extdash}

% Layout: Puts the section titles on left side of page
\reversemarginpar

%
%         PAPER SIZE, PAGE NUMBER, AND DOCUMENT LAYOUT NOTES:
%
% The next \usepackage line changes the layout for CV style section
% headings as marginal notes. It also sets up the paper size as either
% letter or A4. By default, letter was used. If A4 paper is desired,
% comment out the letterpaper lines and uncomment the a4paper lines.
%
% As you can see, the margin widths and section title widths can be
% easily adjusted.
%
% ALSO: Notice that the includefoot option can be commented OUT in order
% to put the PAGE NUMBER *IN* the bottom margin. This will make the
% effective text area larger.
%
% IF YOU WISH TO REMOVE THE ``of LASTPAGE'' next to each page number,
% see the note about the +LP and -LP lines below. Comment out the +LP
% and uncomment the -LP.
%
% IF YOU WISH TO REMOVE PAGE NUMBERS, be sure that the includefoot line
% is uncommented and ALSO uncomment the \pagestyle{empty} a few lines
% below.
%

%% Use these lines for letter-sized paper
\usepackage[paper=a4paper,
            %includefoot, % Uncomment to put page number above margin
            marginparwidth=1in,     % Length of section titles
            marginparsep=.05in,       % Space between titles and text
            margin=0.5in,               % 1 inch margins
            top=1in,
            includemp
            ]{geometry}

%% Use these lines for A4-sized paper
%\usepackage[paper=a4paper,
%            %includefoot, % Uncomment to put page number above margin
%            marginparwidth=30.5mm,    % Length of section titles
%            marginparsep=1.5mm,       % Space between titles and text
%            margin=25mm,              % 25mm margins
%            includemp]{geometry}

%% More layout: Get rid of indenting throughout the entire document
\usepackage{comment}

% ...   
\setlength{\parindent}{0in}

% Provides special list environments and macros to create new ones
\usepackage[shortlabels]{enumitem}

\usepackage[nodayofweek]{datetime}
\newdateformat{mydate}{\twodigit{\THEDAY}{ }\shortmonthname[\THEMONTH], \THEYEAR}


% Simpler bib sections for CV sections
% (thanks to natbib for inspiration)
%
% * For lists of references with hanging indents and no numbers:
%
%   \begin{bibsection}
%       \item ...
%   \end{bibsection}
%
% * For numbered lists of references (with hanging indents):
%
%   \begin{bibenum}
%       \item ...
%   \end{bibenum}
%
%   Note that bibenum numbers continuously throughout. To reset the
%   counter, use
%
%   \restartlist{bibenum}
%
%   at the place where you want the numbering to reset.

\makeatletter
\newlength{\bibhang}
\setlength{\bibhang}{1em}
\newlength{\bibsep}
 {\@listi \global\bibsep\itemsep \global\advance\bibsep by\parsep}
\newlist{bibsection}{itemize}{3}
\setlist[bibsection]{label=,leftmargin=\bibhang,%
        itemindent=-\bibhang,
        itemsep=\bibsep,parsep=\z@,partopsep=0pt,
        topsep=0pt}
\newlist{bibenum}{enumerate}{3}
\setlist[bibenum]{label=[\arabic*],resume,leftmargin={\bibhang+\widthof{[999]}},%
%        itemindent=-\bibhang,
%        itemsep=\bibsep,parsep=\z@,partopsep=0pt,
        topsep=0pt
      }
\let\oldendbibenum\endbibenum
\def\endbibenum{\oldendbibenum\vspace{-.6\baselineskip}}
\let\oldendbibsection\endbibsection
\def\endbibsection{\oldendbibsection\vspace{-.6\baselineskip}}
\makeatother

%% Reference the last page in the page number
%
% NOTE: comment the +LP line and uncomment the -LP line to have page
%       numbers without the ``of ##'' last page reference)
%
% NOTE: uncomment the \pagestyle{empty} line to get rid of all page
%       numbers (make sure includefoot is commented out above)
%
\usepackage{fancyhdr,lastpage}

\newlength{\myoddoffset}
\setlength{\myoddoffset}{\marginparwidth + \marginparsep}

\pagestyle{fancy}
%\pagestyle{empty}      % Uncomment this to get rid of page numbers

\fancyheadoffset[leh,roh]{\marginparsep}
\fancyheadoffset[loh,reh]{\myoddoffset}

\fancyhf{}\renewcommand{\headrulewidth}{1pt}
\fancyhead[L]{\large\textbf{Fahimeh Azari}}

\fancyfootoffset{\marginparsep+\marginparwidth}
\newlength{\footpageshift}
\setlength{\footpageshift}
          {0.5\textwidth+0.5\marginparsep+0.5\marginparwidth-2in}
 
\fancyfoot[C]{}         
\fancyfoot[L]{
	\hspace{\footpageshift}%
	\parbox{4in}{\, \hfill %
		\arabic{page} of \protect\pageref*{LastPage} % +LP
		\hfill \,}
}

\fancyhead[R]{
	Last update: \mydate\today
}
% Finally, give us PDF bookmarks
\usepackage{color,hyperref}
\definecolor{darkblue}{rgb}{0.0,0.0,0.3}
\hypersetup{colorlinks,breaklinks,
            linkcolor=darkblue,urlcolor=darkblue,
            anchorcolor=darkblue,citecolor=darkblue}
\newcommand{\MYhref}[3][blue]{\href{#2}{\color{#1}{#3}}}%
%%%%%%%%%%%%%%%%%%%%%%%% End Document Setup %%%%%%%%%%%%%%%%%%%%%%%%%%%%

%%%%%%%%%%%%%%%%%%%%%%%%%%% Helper Commands %%%%%%%%%%%%%%%%%%%%%%%%%%%%

%%% HEADING AT TOP OF CURRICULUM VITAE

% The title (name) with a horizontal rule under it
% (optional argument typesets an object right-justified across from name
%  as well)
%
% Usage: \makeheading{name}
%        OR
%        \makeheading[right_object]{name}
%
% Place at top of document. It should be the first thing.
% If ``right_object'' is provided in the square-braced optional
% argument, it will be right justified on the same line as ``name'' at
% the top of the CV. For example:
%
%       \makeheading[\emph{Curriculum vitae}]{Your Name}
%
% will put an emphasized ``Curriculum vitae'' at the top of the document
% as a title. Likewise, a picture could be included:
%
%   \makeheading[{\includegraphics[height=1.5in]{my_picture}}]{Your Name}
%
% the picture will be flush right across from the name. For this example
% to work, make sure the extra set of curly braces is included. Also
% makes ure that \usepackage{graphicx} is somewhere in the preamble.
\newcommand{\makeheading}[2][]%
        {\hspace*{-\marginparsep minus \marginparwidth}%
         \begin{minipage}[t]{\textwidth+\marginparwidth+\marginparsep}%
             {\large \bfseries #2 \hfill #1}\\[-0.15\baselineskip]%
                 \rule{\columnwidth}{1pt}%
         \end{minipage}}

%%% SECTION HEADINGS

% The section headings. Flush left in small caps down pseudo-margin.
%
% Usage: \section{section name}
\renewcommand{\section}[1]{\pagebreak[3]%
    \vspace{1.3\baselineskip}%
    \phantomsection\addcontentsline{toc}{section}{#1}%
    \noindent\llap{\scshape\smash{\parbox[t]{\marginparwidth}{\hyphenpenalty=10000\raggedright #1}}}%
    \vspace{-\baselineskip}\par}

%%% LISTS

% This macro alters a list by removing some of the space that follows the list
% (is used by lists below)
\newcommand*\fixendlist[1]{%
    \expandafter\let\csname preFixEndListend#1\expandafter\endcsname\csname end#1\endcsname
    \expandafter\def\csname end#1\endcsname{\csname preFixEndListend#1\endcsname\vspace{-0.6\baselineskip}}}

% These macros help ensure that items in outer-type lists do not get
% separated from the next line by a page break
% (they are used by lists below)
\let\originalItem\item
\newcommand*\fixouterlist[1]{%
    \expandafter\let\csname preFixOuterList#1\expandafter\endcsname\csname #1\endcsname
    \expandafter\def\csname #1\endcsname{\let\oldItem\item\def\item{\pagebreak[2]\oldItem}\csname preFixOuterList#1\endcsname}
    \expandafter\let\csname preFixOuterListend#1\expandafter\endcsname\csname end#1\endcsname
    \expandafter\def\csname end#1\endcsname{\let\item\oldItem\csname preFixOuterListend#1\endcsname}}
\newcommand*\fixinnerlist[1]{%
    \expandafter\let\csname preFixInnerList#1\expandafter\endcsname\csname #1\endcsname
    \expandafter\def\csname #1\endcsname{\let\oldItem\item\let\item\originalItem\csname preFixInnerList#1\endcsname}
    \expandafter\let\csname preFixInnerListend#1\expandafter\endcsname\csname end#1\endcsname
    \expandafter\def\csname end#1\endcsname{\csname preFixInnerListend#1\endcsname\let\item\oldItem}}

% An itemize-style list with lots of space between items
%
% Usage:
%   \begin{outerlist}
%       \item ...    % (or \item[] for no bullet)
%   \end{outerlist}
\newlist{outerlist}{itemize}{3}
    \setlist[outerlist]{label=\enskip\textbullet,leftmargin=*}
    \fixendlist{outerlist}
    \fixouterlist{outerlist}

% An environment IDENTICAL to outerlist that has better pre-list spacing
% when used as the first thing in a \section
%
% Usage:
%   \begin{lonelist}
%       \item ...    % (or \item[] for no bullet)
%   \end{lonelist}
\newlist{lonelist}{itemize}{3}
    \setlist[lonelist]{label=\enskip\textbullet,leftmargin=*,partopsep=0pt,topsep=0pt}
    \fixendlist{lonelist}
    \fixouterlist{lonelist}

% An itemize-style list with little space between items
%
% Usage:
%   \begin{innerlist}
%       \item ...    % (or \item[] for no bullet)
%   \end{innerlist}
\newlist{innerlist}{itemize}{3}
    \setlist[innerlist]{label=\enskip\textbullet,leftmargin=*,parsep=0pt,itemsep=0pt,topsep=0pt,partopsep=0pt}
    \fixinnerlist{innerlist}

% An environment IDENTICAL to innerlist that has better pre-list spacing
% when used as the first thing in a \section
%
% Usage:
%   \begin{loneinnerlist}
%       \item ...    % (or \item[] for no bullet)
%   \end{loneinnerlist}
\newlist{loneinnerlist}{itemize}{3}
    \setlist[loneinnerlist]{label=\enskip\textbullet,leftmargin=*,parsep=0pt,itemsep=0pt,topsep=0pt,partopsep=0pt}
    \fixendlist{loneinnerlist}
    \fixinnerlist{loneinnerlist}

%%% EXTRA SPACE

% To add some paragraph space between lines.
% This also tells LaTeX to preferably break a page on one of these gaps
% if there is a needed pagebreak nearby.
\newcommand{\blankline}{\quad\pagebreak[3]}
\newcommand{\halfblankline}{\quad\vspace{-0.5\baselineskip}\pagebreak[3]}

%%% FORMATTING MACROS

% Provides a linked \doi{#1} that links doi:#1 to http://dx.doi.org/#1
\usepackage{doi}
% To change the text before the DOI, adjust this command
%\renewcommand\doitext{doi:}

% Provides a linked \url{#1} that doesn't require escape characters
\usepackage{url}

% You can adjust the style \url{} uses here:
% (options are: same, rm, sf, tt; defaults to tt)
\urlstyle{same}

% For \email{ADDRESS}, links ADDRESS to the url mailto:ADDRESS
% (uncomment to typeset the e\-/mail address in typewriter font;
%  otherwise, will be typeset in the \urlstyle above)
%\DeclareUrlCommand\emaillink{\urlstyle{tt}}
\providecommand*\emaillink[1]{\nolinkurl{#1}}
\providecommand*\email[1]{\href{mailto:#1}{\emaillink{#1}}}

\providecommand\BibTeX{{B\kern-.05em{\sc i\kern-.025em b}\kern-.08em \TeX}}
\providecommand\Matlab{\textsc{Matlab}}

% Custom hyphenation rules for words that LaTeX has trouble with
\hyphenation{bio-mim-ic-ry bio-in-spi-ra-tion re-us-a-ble pro-vid-er Media-Wiki}

%%%%%%%%%%%%%%%%%%%%%%%% End Helper Commands %%%%%%%%%%%%%%%%%%%%%%%%%%%
\def\longversion{1}  % set to 0 for short version
%%%%%%%%%%%%%%%%%%%%%%%%% Begin CV Document %%%%%%%%%%%%%%%%%%%%%%%%%%%%

\begin{document}
%\makeheading{Sina Sajadmanesh}

\section{Contact Information}

% NOTE: Mind where the & separators and \\ breaks are in the following
%       table. Table is one row made up of three parboxes. The left
%       parbox has address info, the middle parbox has a vertical bar,
%       and the right parbox has phone and electronic contact
%       information.
%
% MACROS: \rcollength is the width of the right column of the table
%             (adjust it to your liking; default is 1.85in).
%         \spacewidth is width of area between left and right boxes.
%
%\newlength{\rcollength}\setlength{\rcollength}{0in}%
\newlength{\spacewidth}\setlength{\spacewidth}{20pt}
%
%\begin{tabular}[t]{@{}p{\textwidth-\rcollength-\spacewidth}@{}p{\spacewidth}@{}p{\rcollength}}%
\begingroup
\setlength{\tabcolsep}{0pt} % Default value: 6pt
\begin{tabu} to \textwidth {X[l]X[r]}
	KU Leuven, BioMechanics (BMe) & (+32) 45-614-22-10~\faPhone\\
	Celestijnenlaan 300 & \hfill{\href{mailto:fahimeh.azari@kuleuven.be}{fahimeh.azari@kuleuven.be}~\faEnvelope} \\  
	box 2422 & \hfill{\href{https://sajadmanesh.com}{https://fahimehazari.github.io}~\faHome} \\
	3001 Leuven &
\end{tabu}
\endgroup
%
%Rue Marconi 19 \hfill (+41) 27-721-77-58~\faPhone\\
%1920 Martigny \hfill{\href{mailto:sajadmanesh@idiap.ch}{sajadmanesh@idiap.ch}~\faEnvelope} \\ 
%Switzerland \hfill{\href{https://sajadmanesh.com}{https://sajadmanesh.com}~\faHome}

%\hfill{GitHub: \href{http://www.github.com/pauljwright}{www.github.com/pauljwright}} }




%\end{tabular}

%%
%% In modern CV's, it seems like ``Objective'' is frowned upon. Instead,
%% incorporate it into a well-constructed cover letter. The ``More
%% information'' can go at the end of the CV, but it should not distract
%% from the section giving references available to contact.
%%
%
% \section{Objective}
%
% Placement in an academic position (i.e., faculty, postdoctoral, or
% research scientist) that allows for advanced research in distributed
% complex adaptive systems (i.e., modeling, analysis, design, and
% verification) with a particular focus on the control of engineered
% agents (e.g., for communications, control, software, electronics, and
% sustainability) and the analysis of biological phenomena (e.g.,
% self-organization, ecological rationality)
% \begin{innerlist}
% \item More information and auxiliary documents can be found at\\\url{http://www.tedpavlic.com/facjobsearch/}
% \end{innerlist}

\if\longversion1
\section{Research Interests}

% My research interests lie at the intersection of privacy, deep learning, and graph analysis. More specifically, I use privacy enhancing technologies, such as differential privacy and federated learning, with graph representation learning algorithms, including graph neural networks, to make them more private, secure, and robust for real-world applications.

Computational Biomechanics, Bone Mechanics, Medical Imaging, Finite Element Analysis
\fi


%%%% EDUCATION %%%%
\section{Education}

\href{https://www.kuleuven.be/english/kuleuven}{\textbf{Katholieke Universiteit Leuven (KU Leuven)}}, Leuven, Belgium, {June 2019 -- Present}
\begin{innerlist}
\item[] Ph.D. in Mechanical Engineering \quad %No GPA
        \begin{innerlist}
        \item[] \textbf{Thesis:} \emph{Mechanical and structural alterations after surgical treatment of knee osteoarthritis}
        \item[] \textbf{Adviser:} Prof.~G. Harry van Lenthe
        \if\longversion1
        %\item[] \textbf{Relevant Courses:} Supervised Machine Learning; Regression and Classification, Deep Learning Specialization, Data Scientist with Python, Machine Learning Scientist with Python, Advanced Topics in Tissue Mechanics
        \fi
        \end{innerlist}

\end{innerlist}

\halfblankline

\href{http://www.en.sharif.edu/}{\textbf{Sharif University of Technology}}, Tehran, Iran, {Sep 2014 -- Sep 2018}
\begin{innerlist}
	\item[] M.Sc. in Mechanical Engineering (Applied Design) \quad  GPA: 18.5 / 20
	\begin{innerlist}
		\item[] \textbf{Thesis:} \emph{Passive Finite Element Model Combined with a Musculoskeletal Model of the Spine to Estimate in vivo Load Sharing in the
L4-L5 Motion Segment}
		\item[] \textbf{Adviser:} Prof.~Navid Arjmand and Prof.~Mohamad Parnianpour
		\if\longversion1
  
		%\item[] \textbf{Relevant Courses:} 
		\fi
	\end{innerlist}
	
\end{innerlist}

\halfblankline

\href{https://aut.ac.ir/en}{\textbf{Amirkabir University of Technology (Tehran Polytechnic)}}, Tehran, Iran, {Sep 2010 -- Sep 2014}
\begin{innerlist}
	\item[] B.Sc. in Biomedical Engineering (Biomechanics) \quad  GPA: 18.91 / 20
	\begin{innerlist}
		\item[] \textbf{Project:} \emph{Design and Implementation of an Elastography Apparatus for Mechanical Characterization of Soft Tissues}
		\item[] \textbf{Adviser:} Prof.~Nasser Fatouraee
		\if\longversion1
		%\item[] \textbf{Relevant Courses:} 
		\fi
	\end{innerlist}
	
\end{innerlist}
\href{https://aut.ac.ir/en}{\textbf{Amirkabir University of Technology (Tehran Polytechnic)}}, Tehran, Iran, {Sep 2011 -- Sep 2014}
\begin{innerlist}
	\item[] Passing 90 credits in Industrial Engineering, Studying two majors at the same time (Double degree) \quad  GPA: 17.71 / 20

	\end{innerlist}


\section{Research Experience}


\textbf{Research Assistant}, {August 2018 -- June 2019}
\begin{innerlist}
    \item[] \href{https://www.mech.kuleuven.be/en/bme}{BioMechanics (BMe), Bone Group}, \textbf{\href{https://idiap.ch}{KU Leuven}}, Leuven, Belgium
    \if\longversion1
    \begin{innerlist}
    	\item Developing a computational study of line‐to‐line versus undersized cementing techniques of short‐stem total hip arthroplasty with experimental verification.
    \end{innerlist}
  	\fi
\end{innerlist}

\halfblankline

\textbf{PhD Researcher}, {June 2019 -- Present}
\begin{innerlist}
    \item[] \href{https://www.idiap.ch/en/scientific-research/social-computing/index_html}{BioMechanics (BMe), Bone Group}, \textbf{\href{https://idiap.ch}{KU Leuven}}, Leuven, Belgium
    \if\longversion1
    \begin{innerlist}
    	\item Visualization and quantification of bone microstructure in knee osteoarthritis. 
    \end{innerlist}
  	\fi
\end{innerlist}

\halfblankline


\textbf{Research Assistant}, {Sep 2016 -- Sep 2017}
\begin{innerlist}
	\item[] {Biomechanics Lab}, \href{http://www.en.sharif.edu/}{\textbf{Sharif University of Technology}}, Tehran, Iran
	\if\longversion1
	\begin{innerlist}
		\item  Finite element analysis based on CT scans modeling; developing a combined passive and active musculoskeletal model study to estimate L4-L5 load sharing based.
	\end{innerlist}
	\fi
\end{innerlist}
\textbf{Member of Research and Development (R\&D) group}, {Sep 2017 -- August 2018}
\begin{innerlist}
	\item[] {R\&D group}, \href{https://www.zoominfo.com/c/dideban-tajhiz-alborz-co/437248784}{Dideban Tajhiz Alborz Co.}, Tehran, Iran
	\end{innerlist}



\section{Teaching Experience}

\textbf{Teaching Assistant}, {October 2022}
\begin{innerlist}
	\item[] \href{https://www.mech.kuleuven.be/en}{Department of Mechanical engineering}, \href{https://www.kuleuven.be/english/kuleuven}{\textbf{KU Leuven}}, Leuven, Belgium
	\begin{innerlist}
		\item[] \textbf{Course:} \href{https://onderwijsaanbod.kuleuven.be//syllabi/e/H0N47AE.htm#activetab=doelstellingen_idp12075776} {\textbf{Advanced Tissue Mechanics}}
  \begin{innerlist}
      	  \item   Tutoring Materialise software packages (mimics and 3-matic) for developing subject-specific finite element models of implanted femora to identify critical locations prone to failure under physiological loading conditions. 
    \end{innerlist}
\end{innerlist}

		
	\end{innerlist}

\halfblankline


\textbf{Teaching Assistant}, {November 2021}
\begin{innerlist}
	\item[] \href{https://www.mech.kuleuven.be/en}{Department of Mechanical engineering}, \href{https://www.kuleuven.be/english/kuleuven}{\textbf{KU Leuven}}, Leuven, Belgium
	\begin{innerlist}
		\item[] \textbf{Course:} \href{https://onderwijsaanbod.kuleuven.be//syllabi/e/H0N44AE.htm#activetab=doelstellingen_idp65744} {\textbf{Numerical Modelling in Biomedical Engineering}}
    \begin{innerlist}
      	  \item   Tutoring students in developing a python tool for computed tomography-based structural rigidity analysis (CTRA) to predict and monitor fracture risk associated with metastatic bone lesions. 
 
    \end{innerlist}

		
	\end{innerlist}
\end{innerlist}

\halfblankline



\textbf{Teaching Assistant}, {October 2021, October 2022}
\begin{innerlist}
	\item[] \href{https://www.mech.kuleuven.be/en}{Department of Mechanical engineering}, \href{https://www.kuleuven.be/english/kuleuven}{\textbf{KU Leuven}}, Leuven, Belgium
	\begin{innerlist}
		\item[] \textbf{Course:} \href{https://onderwijsaanbod.kuleuven.be//syllabi/e/H06X2AE.htm#activetab=doelstellingen_idp59688} {\textbf{Advanced Study Topics in Musculoskeletal Biomechanics}}
    \begin{innerlist}
      	  \item   Tutoring Materialise software packages (mimics and 3-matic) and ABAQUS to quantify bone tissue modulus in CT-based finite element models of the femur using three-point bending and beam theory. 
 
    \end{innerlist}
	
	\end{innerlist}
\end{innerlist}

\halfblankline

\begin{comment}
\textbf{Teaching Assistant}, {October 2020}
\begin{innerlist}
	\item[] \href{https://www.mech.kuleuven.be/en}{Department of Mechanical engineering}, \href{https://www.kuleuven.be/english/kuleuven}{\textbf{KU Leuven}}, Leuven, Belgium
	\begin{innerlist}
		\item[] \textbf{Course:} \href{https://onderwijsaanbod.kuleuven.be//syllabi/e/H06X2AE.htm#activetab=doelstellingen_idp596880} {\textbf{Advanced Study Topics in Musculoskeletal Biomechanics}}
		
	\end{innerlist}
\end{innerlist}

\halfblankline
\end{comment}

\textbf{Teaching Assistant}, {Fall 2016, Fall 2015}
\begin{innerlist}
	\item[] \href{http://library.sharif.ir/web/mech/20}{Department of Mechanical engineering}, \href{https://en.sharif.edu/}{\textbf{Sharif University of Technology}}, Tehran, Iran
	\begin{innerlist}
		\item[] \textbf{Course:}  {\textbf{Statics}}
		
	\end{innerlist}
\end{innerlist}

\halfblankline

\textbf{Teaching Assistant}, {Fall 2015}
\begin{innerlist}
	\item[] \href{http://math.sharif.ir/}{Department of Mathematics}, \href{https://en.sharif.edu/}{\textbf{Sharif University of Technology}}, Tehran, Iran
	\begin{innerlist}
		\item[] \textbf{Tutoring MATLAB software} to Bachelor Students
		
	\end{innerlist}
\end{innerlist}

\halfblankline

\textbf{Teaching Assistant}, {Spring 2016, Spring 2015}
\begin{innerlist}
	\item[] \href{http://library.sharif.ir/web/mech/20}{Department of Mechanical engineering}, \href{https://en.sharif.edu/}{\textbf{Sharif University of Technology}}, Tehran, Iran
	\begin{innerlist}
		\item[] \textbf{Course:}  {\textbf{Material Science}}
		
	\end{innerlist}
\end{innerlist}

\halfblankline

\section{Industrial Experience}

\textbf{Content Creator}, {March 2021-Present}
\begin{innerlist}
	\item[] \href{http://ictic.sharif.ir}{Udemy}, Online platform
	\if\longversion1
	\begin{innerlist}
		\item Developing and producing video and written content for online courses on Materialise software packages (mimics \& 3-matic). 
	\end{innerlist}
	\fi
\end{innerlist}

\textbf{Beta Tester}, {June 2020}
\begin{innerlist}
	\item[] \href{http://ictic.sharif.ir}{Materialise Co.}, Leuven, Belgium
	\if\longversion1
	\begin{innerlist}
		\item Testing and providing feedback on the functionality and usability of the upcoming release of the company's materials software (version 23).
	\end{innerlist}
	\fi
\end{innerlist}


 
%
% % Add a little space to nudge next ``Ref'd Journal Publications'' marginpar
% % down to make room for tall ``Submitted Journal Publications''
% % marginpar. If there are enough submitted journal publications, this
% % space will not be needed (and should be removed).
% \vspace{0.1in}

\section{Publications}

\begin{bibenum}
	\item{} \textbf{Fahimeh Azari}, William Colyn, Johan Bellemans, Lennart Scheys, G. Harry van Lenthe\\
	\href{https://scholar.google.com/citations?view_op=view_citation&hl=en&user=C-ZuElMAAAAJ&citation_for_view=C-ZuElMAAAAJ:qjMakFHDy7sC}{\textbf{In the end-stage knee osteoarthritis the subchondral bone microarchitecture of the tibial plateau is correlated to that of the distal femur}}\\
	\textit{27th Congress of the European Society of Biomechanics}, June 2022, Porto, Portugal

	\item{} William Colyn, \textbf{Fahimeh Azari}, Johan Bellemans, G. Harry van Lenthe, Lennart Scheys\\
	\href{https://scholar.google.com/citations?view_op=view_citation&hl=en&user=C-ZuElMAAAAJ&citation_for_view=C-ZuElMAAAAJ:qjMakFHDy7sC}{\textbf{Microstructural adaptations of the subchondral tibial bone are related to the mechanical axis deviation in end-stage varus osteoarthritic knees}}\\
	\textit{OARSI 2022 World Congress on Osteoarthritis}, April 2022, Berlin, Germany


	\item{} \textbf{Fahimeh Azari}, Amelie Sas, Karl P Kutzner, Andreas Klockow, Thierry Scheerlinck, G Harry van Lenthe\\
	\href{https://scholar.google.com/citations?view_op=view_citation&hl=en&user=C-ZuElMAAAAJ&citation_for_view=C-ZuElMAAAAJ:9yKSN-GCB0IC7}{\textbf{Cemented short-stem total hip arthroplasty appears promising in patients with poor bone quality}}\\
	\textit{26th Congress of the European Society of Biomechanics}, August 2021, Milan, Italy

	\item{} \textbf{Fahimeh Azari}, Amelie Sas, Karl P Kutzner, Andreas Klockow, Thierry Scheerlinck, G Harry van Lenthe\\
	\href{https://onlinelibrary.wiley.com/doi/full/10.1002/jor.24887}{\textbf{Cemented short‐stem total hip arthroplasty: Characteristics of line‐to‐line versus undersized cementing techniques using a validated CT‐based finite element analysis}}\\
	\textit{Journal of Orthopaedic Research®}, September 2021
	
		\item{} Chadapa Rungruangbaiyok, \textbf{Fahimeh Azari}, G Harry van Lenthe, Jos Vander Sloten, Boonsin Tangtrakulwanich, Surapong Chatpun\\
	\href{https://link.springer.com/article/10.1007/s40846-021-00607-1}{\textbf{Finite element investigation of fracture risk under postero-anterior mobilization on a lumbar bone in elderly with and without osteoporosis}}\\
	\textit{Journal of Medical and Biological Engineering}, June 2021

	\item{} \textbf{Fahimeh Azari}, Navid Arjmand, Aboulfazl Shirazi-Adl, Shima Rahimi-Moghaddam\\
	\href{https://www.sciencedirect.com/science/article/pii/S0021929017302270}{\textbf{A combined passive and active musculoskeletal model study to estimate L4-L5 load sharing}}\\
        \textit{Journal of biomechanics}, March 2018
	
\end{bibenum}
	
%	\item \textbf{ReachMD},  {\href{https://reachmd.com/news/if-you-are-what-you-eat-regional-cuisines-have-a-major-impact-on-health/1306703/}{If you are what you eat: regional cuisines have a major impact on health}}, {4 Nov 2016}

\section{Talks and \\Presentations}

\textbf{In the end-stage knee osteoarthritis the subchondral bone microarchitecture of the tibial plateau is correlated to that of the distal femur}
\begin{innerlist}
	\item[] 27th Congress of the European Society of Biomechanics, June 2022, Porto, Portugal
\end{innerlist}

\halfblankline

\textbf{Microstructural adaptations of the subchondral tibial bone are related to the mechanical axis deviation in end-stage varus osteoarthritic knees}
\begin{innerlist}
	\item[] OARSI 2022 World Congress on Osteoarthritis, April 2022, Berlin, Germany
\end{innerlist}

\halfblankline

\textbf{Cemented short-stem total hip arthroplasty appears promising in patients with poor bone quality}
\begin{innerlist}
	\item[] 26th Congress of the European Society of Biomechanics (remote), August 2021, Milan, Italy
\end{innerlist}

\halfblankline
\section{Honors\\and Awards}
\begin{innerlist}
\item
\textbf{PhD research assistantship}, 
{Mechanical Engineering, The Pennsylvania State University}, 
{2018} (declined)

\item
\textbf{PhD studentship}
{Institute of Bioengineering in the School of Engineering,  École Polytechnique Fédérale de Lausanne (EPFL)},
{2018} (declined)
\item
\textbf{Membership of}
{National Elites Foundation},
{Iran},
{2010-2018}
\item
\textbf{Fellowship of}
{National Elites Foundation},
{Iran},
{2014-2018}
\item
\textbf{Exceptional Talents Fellowship}
{ in Biomedical Engineering Department,},
{Amirkabir University of Technology, Iran},
{2014}
\item
\textbf{Membership of}
{Exceptional Talent Center,},
{Amirkabir University of Technology, Iran},
{2010-2014}
\item
\textbf{Ranked 1st}
{in cum. GPA among undergraduate biomedical engineering students},
{Amirkabir University of Technology},
{2010-2014}
\end{innerlist}

\if\longversion1
\section{Technical Skills} 

\textit
{Programming Languages:}\\
{Python, R}

\halfblankline

%\textit
%{Machine Learning \& Data Science:}\\
%{PyTorch, PyTorch-Geometric, PyTorch-Lightning, Tensorflow, Scikit-Learn, Pandas}

%\halfblankline

\textit
{Operating Systems:}\\
{Linux, Windows}

\halfblankline

\textit
{Analysis/Simulation Tools:}\\
{MATLAB, Neural Network in MATLAB, Identification System in MATLAB, Simulink, Business Process Modeling in Arena, ABAQUS}

\halfblankline

\textit
{Applications Software:}\\
{CATIA, SolidWorks, AutoCAD, Mimics, 3-matic, Geomagic, Hypermesh, OpenSim, AnyBody}

\halfblankline
\fi

\section{References} Available upon request
%\textbf{Prof. G.Harry van Lenthe}, BioMechanics division (BMe), KU Leuven

%\enskip\begin{tabular}{p{6cm}l}
	%Website: \href{https://harryvanlenthe.com}{https://harryvanlenthe.com} &
	%Email: \href{mailto:harry.vanlenthe@kuleuven.be}%{harry.vanlenthe@kuleuven.be}\\
%\end{tabular}

%\halfblankline

%\textbf{Prof. Navid Arjmand}, Sharif University of Technology

%\enskip\begin{tabular}{p{6cm}l}
	%Website: \href{http://sharif.ir/~arjmand}%{http://sharif.ir/\string~arjmand} &
	%Email: \href{mailto:arjmand@sharif.edu}{arjmand@sharif.edu}\\
%\end{tabular} 

\halfblankline
\end{document}

%%%%%%%%%%%%%%%%%%%%%%%%%% End CV Document %%%%%%%%%%%%%%%%%%%%%%%%%%%%%

%----------------------------------------------------------------------%
% The following is copyright and licensing information for
% redistribution of this LaTeX source code; it also includes a liability
% statement. If this source code is not being redistributed to others,
% it may be omitted. It has no effect on the function of the above code.
%----------------------------------------------------------------------%
% Copyright (c) 2007, 2008, 2009, 2010, 2011 by Theodore P. Pavlic
%
% Unless otherwise expressly stated, this work is licensed under the
% Creative Commons Attribution-Noncommercial 3.0 United States License. To
% view a copy of this license, visit
% http://creativecommons.org/licenses/by-nc/3.0/us/ or send a letter to
% Creative Commons, 171 Second Street, Suite 300, San Francisco,
% California, 94105, USA.
%
% THE SOFTWARE IS PROVIDED "AS IS", WITHOUT WARRANTY OF ANY KIND, EXPRESS
% OR IMPLIED, INCLUDING BUT NOT LIMITED TO THE WARRANTIES OF
% MERCHANTABILITY, FITNESS FOR A PARTICULAR PURPOSE AND NONINFRINGEMENT.
% IN NO EVENT SHALL THE AUTHORS OR COPYRIGHT HOLDERS BE LIABLE FOR ANY
% CLAIM, DAMAGES OR OTHER LIABILITY, WHETHER IN AN ACTION OF CONTRACT,
% TORT OR OTHERWISE, ARISING FROM, OUT OF OR IN CONNECTION WITH THE
% SOFTWARE OR THE USE OR OTHER DEALINGS IN THE SOFTWARE.
%----------------------------------------------------------------------%
